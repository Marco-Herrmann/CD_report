% !TeX spellcheck = en_US
\documentclass[a4paper,10pt,english]{article}
\usepackage[
left=25mm,
right=25mm,
]{geometry}

\usepackage{babel}
\usepackage[T1]{fontenc}
\usepackage{lmodern,textcomp,latexsym}
\usepackage{amsthm,amssymb,amsmath,amsfonts,dsfont,mathrsfs,bbm,bm,mathtools}
\usepackage{graphicx,psfrag,xcolor,rotate}
\usepackage{paralist,threeparttable}
\usepackage[footnotesize]{caption}
\usepackage{dsfont,booktabs,threeparttable}
\usepackage{longtable,booktabs}

%% tikz & pgfplots
\usepackage{tikz}
\usetikzlibrary{calc,patterns,positioning,external} % TikZ-Libraries
\tikzexternalize[prefix=tikz/] % prefix for externalize

\usepackage{pgfplots}
\usepgfplotslibrary{groupplots}
\usepackage{pgfplotstable}
\graphicspath{{graphics/}} %Setting the graphicspath

% settings for PGFPlots
\pgfplotsset{
    compat=1.18, 
    every axis plot/.append style={line width=0.4mm}, 
}

% more pacakges
\usepackage{ifthen}
\usepackage{array}
\usepackage{cite}

% format section headers
\usepackage{titlesec}

\titleformat*{\section}{\large\bfseries}
\titleformat*{\subsection}{\normalsize\bfseries}
\titleformat*{\paragraph}{\normalsize\bfseries}
\titleformat*{\subparagraph}{\normalsize\bfseries}

% blue text for review
\newcommand{\textblue}[1]{\textcolor{blue}{#1}}

% \DeclareCaptionStyle{table_new}{font={footnotesize},justification=raggedright}
% \captionsetup[table]{style=table_new}
% \setdefaultenum{1)}{\theenumi.1)}{}{}

\theoremstyle{plain}
\newtheorem{theorem}{Theorem}
\newtheorem{proposition}{Proposition}
\newtheorem{remark}[theorem]{Remark}

%\newtheoremstyle{bis}% name
%{}% Space above
%{}% Space below
%{\itshape}% Body font
%{}% Indent amount
%{\bfseries}% Theorem head font
%{.}% Punctuation after theorem head
%{ }% Space after theorem head
%{\thmname{#1}\thmnumber{ #2}$^{\prime\prime}$\thmnote{ {\normalfont(#3)}}}% Theorem head spec (can be left empty, meaning ‘normal’)
%\theoremstyle{bis}
%\newtheorem{theoremBis}{Theorem}

\theoremstyle{definition}
\newtheorem{definition}{Definition}
\newtheorem{corollary}{Corollary}

\newcommand{\Pone}{P^1}
\newcommand{\vPone}{{\mathbf P}^1}
\newcommand{\Ptwo}{{P}^2}
\newcommand{\vPtwo}{{\mathbf P}^2}

\newcommand{\triplecontraction}{\mathop{{:\kern -3pt\cdot}}}


% general commands

\newcommand{\interiorProd}{%
	\kern0.5pt
	\raisebox{0.85\depth}{\scalebox{1}[-1.2]{$\lnot$}}
	\kern1.5pt
}

% Commands/Abbreviations
\newcommand{\textAND}{\quad\text{and}\quad}
% \newcommand{\vect}{\mathop{\mathrm{Vect}}}

\newcommand{\mperp}{{m_{\!\perp \phantom{\!\parallel}}\!\!}}
\newcommand{\mparallel}{{m_{\parallel}}}
\newcommand{\Mparallel}{{M_{\parallel}}}
\newcommand{\divergencep}[1]{\mathop{\mathrm{div}_{\parallel}^{#1}}}

\newcommand{\Div}{\mathop{\mathrm{Div}}}
\newcommand{\ip}[2]{\langle#1,#2\rangle}
\newcommand{\bbc}{\mathbbm{c}}
\newcommand{\bbd}{\mathbbm{d}}
\newcommand{\bbP}{\mathbb{P}}
\newcommand{\bbF}{\mathbb{F}}
\newcommand{\bbs}{\mathbbm{s}}

% \newcommand{\diff}[1][]{\mathrm{d}#1}
\newcommand{\Diff}[1][]{\mathrm{D}#1}
\newcommand{\dt}{\diff{t}}
\newcommand{\dx}{\diff{x}}
\newcommand{\dth}{\diff{\theta}}
\newcommand{\dq}{\diff{\vq}}
\newcommand{\du}{\diff{\vu}}
\newcommand{\dP}{\diff{\vP}}
\newcommand{\dF}{\diff{\vF}}
\newcommand{\deta}{\diff{\eta}}
% \newcommand{\pd}[2]{\frac{\partial #1}{\partial #2}}
% \newcommand{\td}[2]{\frac{\diff #1}{\diff #2}}
\newcommand{\argmin}{\mathop{\mathrm{argmin}}}
\newcommand{\Upr}{{\mathop{\mathrm{Upr}}}}
\newcommand{\Sgn}{{\mathop{\mathrm{Sgn}}}}
\newcommand{\h}{\mathop{\mathrm{H}}}
\newcommand{\prox}{{\mathop{\mathrm{prox}}}}
\newcommand{\abs}{\mathop{\mathrm{abs}}}
% \newcommand{\T}{^{\mathop{\mathrm{T}}}}
\newcommand{\nT}{^{\mathop{\mathrm{-T}}}}
% \newcommand{\diag}{{\mathop{\mathrm{diag}}}}
% \newcommand{\tr}{\mathop{\mathrm{tr}}}

% \newcommand{\mR}{\mathbb{R}}
\newcommand{\mE}{\mathbb{E}}
\newcommand{\mN}{\mathbb{N}}
\newcommand{\mQ}{\mathbb{Q}}

%% Mathematical symbols
% the real numbers
\newcommand{\mR}{\mathbb{R}}
% the three-dimensional Euclidian space
\newcommand{\Eucl}{\mathbb{E}^3}
% the set of all second order Euclidean tensors
\newcommand{\EuclSecTens}{\operatorname{L}(\Eucl; \Eucl)}
% the set of permutations
\newcommand{\perm}{\mathrm{Perm}}
% the general linear group
\newcommand{\GL}{G\!L}
\newcommand{\gl}{\mathfrak{gl}}
% the orthogonal group
\newcommand{\Orth}{O}
% the special orthogonal group
\newcommand{\SO}{S\!O}
\newcommand{\so}{\mathfrak{so}}
\let\so\undefined
\newcommand{\so}{\mathfrak{so}}
% the special Euclidean group
\newcommand{\SE}{{SE}}
\newcommand{\se}{\mathfrak{se}}

%\newcommand{\exp}{\operatorname{exp}}
%\newcommand{\log}{\operatornam{log}}
\newcommand{\Skw}{\operatorname{Skw}}
\newcommand{\dexp}{\operatorname{dexp}}
\newcommand{\Exp}{\operatorname{Exp}}
\newcommand{\Log}{\operatorname{Log}}
\newcommand{\Ad}[1]{\operatorname{Ad}_{#1}}
\newcommand{\ad}[1]{\operatorname{ad}_{#1}}
\newcommand{\tr}{\operatorname{tr}}

% matrix transpose
\newcommand{\T}{^{\!\mathrm{T}}}
% exterior derivative
\newcommand{\ed}{d}
% differential of a function
\newcommand{\DD}{D}
% integration d
\newcommand{\dd}{\mathrm{d}}
% set of smooth functions
\newcommand{\funct}{\mathrm{C}^\infty}
% set of smooth vector fields
\newcommand{\vect}{\mathrm{Vect}}
% set of smooth covector fields
\newcommand{\covect}{\mathrm{\Omega}^1}
% differential measure
\newcommand{\diff}[1][]{\mathrm{d}#1}

% induced basis on the tangent space
\newcommand{\e}[1]{\frac{\partial}{\partial #1}}

% Lie derivative
\newcommand{\lie}{\mathrm{L}}
% partial derivative
\newcommand{\pd}[2]{\frac{\partial #1}{\partial #2}}
% total derivative
\newcommand{\td}[2]{\frac{\mathrm{d} #1}{\mathrm{d} #2}}

% identity map
\newcommand{\id}{\mathrm{id}}
% diagonal matrix
\DeclareMathOperator{\diag}{\operatorname{diag}}
% projection map
\newcommand{\pr}{\mathrm{pr}}
\DeclareMathOperator{\sgn}{sgn}

% the zero vector
\newcommand{\vzero}{\mathbf{0}}
\newcommand{\veczero}{\symsfup{0}}

% span
\DeclareMathOperator{\Span}{span}

% capital kappa
\newcommand{\Kappa}{\mathrm{K}}
\newcommand{\vKa}{{\bm{\Kappa}}}


%% bold letters
%bold greek
\newcommand{\val}{{\bm{\alpha}}}
\newcommand{\vbe}{{\bm{\beta}}}
\newcommand{\vga}{{\bm{\gamma}}}
\newcommand{\vde}{{\bm{\delta}}}
\newcommand{\vep}{{\bm{\epsilon}}}
\newcommand{\vze}{{\bm{\zeta}}}
\newcommand{\vet}{{\bm{\eta}}}
\newcommand{\vth}{{\bm{\theta}}}
\newcommand{\vio}{{\bm{\iota}}}
\newcommand{\vka}{{\bm{\kappa}}}
\newcommand{\vla}{{\bm{\lambda}}}
\newcommand{\vmu}{{\bm{\mu}}}
\newcommand{\vnu}{{\bm{\nu}}}
\newcommand{\vxi}{{\bm{\xi}}}
\newcommand{\vpi}{{\bm{\pi}}}
\newcommand{\vrh}{{\bm{\rho}}}
\newcommand{\vsi}{{\bm{\sigma}}}
\newcommand{\vta}{{\bm{\tau}}}
\newcommand{\vup}{{\bm{\upsilon}}}
%\newcommand{\vph}{{\bm{\varphi}}}
\newcommand{\vph}{{\bm{\phi}}} % springer template version
\newcommand{\vch}{{\bm{\chi}}}
\newcommand{\vps}{{\bm{\psi}}}
\newcommand{\vom}{{\bm{\omega}}}

\newcommand{\vvep}{{\bm{\varepsilon}}}
\newcommand{\vvth}{{\bm{\vartheta}}}
\newcommand{\vvrh}{{\bm{\varrho}}}
\newcommand{\vvpi}{{\bm{\varpi}}}
\newcommand{\vvsi}{{\bm{\varsigma}}}
%\newcommand{\vph}{{\bm{\phi}}}
\newcommand{\vvph}{{\bm{\varphi}}} % springer template version

% bold capital greek
%\newcommand{\vGa}{{\bm{\Gamma}}}
%\newcommand{\vDe}{{\bm{\Delta}}}
%\newcommand{\vTh}{{\bm{\Theta}}}
%\newcommand{\vLa}{{\bm{\Lambda}}}
%\newcommand{\vXi}{{\bm{\Xi}}}
%\newcommand{\vPi}{{\bm{\Pi}}}
%\newcommand{\vSi}{{\bm{\Sigma}}}
%\newcommand{\vUp}{{\bm{\Upsilon}}}
%\newcommand{\vPh}{{\bm{\Phi}}}
%\newcommand{\vPs}{{\bm{\Psi}}}
%\newcommand{\vOm}{{\bm{\Omega}}}
%
\newcommand{\vGa}{{\bm{\Gamma}}}
\newcommand{\vDe}{{\bm{\Delta}}}
\newcommand{\vTh}{{\bm{\Theta}}}
\newcommand{\vLa}{{\bm{\Lambda}}}
\newcommand{\vXi}{{\bm{\Xi}}}
\newcommand{\vPi}{{\bm{\Pi}}}
\newcommand{\vSi}{{\bm{\Sigma}}}
\newcommand{\vUp}{{\bm{\Upsilon}}}
\newcommand{\vPh}{{\bm{\Phi}}}
\newcommand{\vPs}{{\bm{\Psi}}}
\newcommand{\vOm}{{\bm{\Omega}}}

% cursive upright bold capital symbols
% \newcommand{\vGa}{{\bm{\varGamma}}}
% \newcommand{\vDe}{{\bm{\varDelta}}}
% \newcommand{\vTh}{{\bm{\varTheta}}}
% \newcommand{\vLa}{{\bm{\varLambda}}}
% \newcommand{\vXi}{{\bm{\varXi}}}
% \newcommand{\vPi}{{\bm{\varPi}}}
% \newcommand{\vSi}{{\bm{\varSigma}}}
% \newcommand{\vUp}{{\bm{\varUpsilon}}}
% \newcommand{\vPh}{{\bm{\varPhi}}}
% \newcommand{\vPs}{{\bm{\varPsi}}}
% \newcommand{\vOm}{{\bm{\varOmega}}}

%capital greek slanted, OHNE amsmath-package
%\newcommand{\iGa}{\mathnormal{\Gamma}}
%\newcommand{\iDe}{\mathnormal{\Delta}}
%\newcommand{\iTh}{\mathnormal{\Theta}}
%\newcommand{\iLa}{\mathnormal{\Lambda}}
%\newcommand{\iXi}{\mathnormal{\Xi}}
%\newcommand{\iPi}{\mathnormal{\Pi}}
%\newcommand{\iSi}{\mathnormal{\Sigma}}
%\newcommand{\iUp}{\mathnormal{\Upsilon}}
%\newcommand{\iPh}{\mathnormal{\Phi}}
%\newcommand{\iPs}{\mathnormal{\Psi}}
%\newcommand{\iOm}{\mathnormal{\Omega}}

%capital greek slanted, MIT amsmath-package
\newcommand{\iGa}{\varGamma}
\newcommand{\iDe}{\varDelta}
\newcommand{\iTh}{\varTheta}
\newcommand{\iLa}{\varLambda}
\newcommand{\iXi}{\varXi}
\newcommand{\iPi}{\varPi}
\newcommand{\iSi}{\varSigma}
\newcommand{\iUp}{\varUpsilon}
\newcommand{\iPh}{\varPhi}
\newcommand{\iPs}{\varPsi}
\newcommand{\iOm}{\varOmega}

%bold latin
\newcommand{\va}{{\bm{a}}}
\newcommand{\vb}{{\bm{b}}}
\newcommand{\vc}{{\bm{c}}}
\newcommand{\vd}{{\bm{d}}}
\newcommand{\ve}{{\bm{e}}}
\newcommand{\vf}{{\bm{f}}}
\newcommand{\vg}{{\bm{g}}}
\newcommand{\vh}{{\bm{h}}}
\newcommand{\vi}{{\bm{i}}}
\newcommand{\vj}{{\bm{j}}}
\newcommand{\vk}{{\bm{k}}}
\newcommand{\vl}{{\bm{l}}}
\newcommand{\vm}{{\bm{m}}}
\newcommand{\vn}{{\bm{n}}}
\newcommand{\vo}{{\bm{o}}}
\newcommand{\vp}{{\bm{p}}}
\newcommand{\vq}{{\bm{q}}}
\newcommand{\vr}{{\bm{r}}}
\newcommand{\vs}{{\bm{s}}}
\newcommand{\vt}{{\bm{t}}}
\newcommand{\vu}{{\bm{u}}}
\newcommand{\vv}{{\bm{v}}}
\newcommand{\vw}{{\bm{w}}}
\newcommand{\vx}{{\bm{x}}}
\newcommand{\vy}{{\bm{y}}}
\newcommand{\vz}{{\bm{z}}}
% \newcommand{\eins}{{\bm{1}}}
\newcommand{\eins}{{\mathds{1}}}

%bold capital latin
\newcommand{\vA}{{\bm{A}}}
\newcommand{\vB}{{\bm{B}}}
\newcommand{\vC}{{\bm{C}}}
\newcommand{\vD}{{\bm{D}}}
\newcommand{\vE}{{\bm{E}}}
\newcommand{\vF}{{\bm{F}}}
\newcommand{\vG}{{\bm{G}}}
\newcommand{\vH}{{\bm{H}}}
\newcommand{\vI}{{\bm{I}}}
\newcommand{\vJ}{{\bm{J}}}
\newcommand{\vK}{{\bm{K}}}
\newcommand{\vL}{{\bm{L}}}
\newcommand{\vM}{{\bm{M}}}
\newcommand{\vN}{{\bm{N}}}
\newcommand{\vO}{{\bm{O}}}
\newcommand{\vP}{{\bm{P}}}
\newcommand{\vQ}{{\bm{Q}}}
\newcommand{\vR}{{\bm{R}}}
\newcommand{\vS}{{\bm{S}}}
\newcommand{\vT}{{\bm{T}}}
\newcommand{\vU}{{\bm{U}}}
\newcommand{\vV}{{\bm{V}}}
\newcommand{\vW}{{\bm{W}}}
\newcommand{\vX}{{\bm{X}}}
\newcommand{\vY}{{\bm{Y}}}
\newcommand{\vZ}{{\bm{Z}}}

%calligraphic
\newcommand{\cA}{\mathcal{A}}
\newcommand{\cB}{\mathcal{B}}
\newcommand{\cC}{\mathcal{C}}
\newcommand{\cD}{\mathcal{D}}
\newcommand{\cE}{\mathcal{E}}
\newcommand{\cF}{\mathcal{F}}
\newcommand{\cG}{\mathcal{G}}
\newcommand{\cH}{\mathcal{H}}
\newcommand{\cI}{\mathcal{I}}
\newcommand{\cJ}{\mathcal{J}}
\newcommand{\cK}{\mathcal{K}}
\newcommand{\cL}{\mathcal{L}}
\newcommand{\cM}{\mathcal{M}}
\newcommand{\cN}{\mathcal{N}}
\newcommand{\cO}{\mathcal{O}}
\newcommand{\cP}{\mathcal{P}}
\newcommand{\cQ}{\mathcal{Q}}
\newcommand{\cR}{\mathcal{R}}
\newcommand{\cS}{\mathcal{S}}
\newcommand{\cT}{\mathcal{T}}
\newcommand{\cU}{\mathcal{U}}
\newcommand{\cV}{\mathcal{V}}
\newcommand{\cW}{\mathcal{W}}
\newcommand{\cX}{\mathcal{X}}
\newcommand{\cY}{\mathcal{Y}}
\newcommand{\cZ}{\mathcal{Z}}

%fraktur
\newcommand{\frA}{\mathfrak{A}}
\newcommand{\frB}{\mathfrak{B}}
\newcommand{\frC}{\mathfrak{C}}
\newcommand{\frD}{\mathfrak{D}}
\newcommand{\frE}{\mathfrak{E}}
\newcommand{\frF}{\mathfrak{F}}
\newcommand{\frG}{\mathfrak{G}}
\newcommand{\frH}{\mathfrak{H}}
\newcommand{\frI}{\mathfrak{I}}
\newcommand{\frJ}{\mathfrak{J}}
\newcommand{\frK}{\mathfrak{K}}
\newcommand{\frL}{\mathfrak{L}}
\newcommand{\frM}{\mathfrak{M}}
\newcommand{\frN}{\mathfrak{N}}
\newcommand{\frO}{\mathfrak{O}}
\newcommand{\frP}{\mathfrak{P}}
\newcommand{\frQ}{\mathfrak{Q}}
\newcommand{\frR}{\mathfrak{R}}
\newcommand{\frS}{\mathfrak{S}}
\newcommand{\frT}{\mathfrak{T}}
\newcommand{\frU}{\mathfrak{U}}
\newcommand{\frV}{\mathfrak{V}}
\newcommand{\frW}{\mathfrak{W}}
\newcommand{\frX}{\mathfrak{X}}
\newcommand{\frY}{\mathfrak{Y}}
\newcommand{\frZ}{\mathfrak{Z}}

\newcommand{\fra}{\mathfrak{a}}
\newcommand{\frb}{\mathfrak{b}}
\newcommand{\frc}{\mathfrak{c}}
\newcommand{\frd}{\mathfrak{d}}
\newcommand{\fre}{\mathfrak{e}}
\newcommand{\frf}{\mathfrak{f}}
\newcommand{\frg}{\mathfrak{g}}
\newcommand{\frh}{\mathfrak{h}}
\newcommand{\fri}{\mathfrak{i}}
\newcommand{\frj}{\mathfrak{j}}
\newcommand{\frk}{\mathfrak{k}}
\newcommand{\frl}{\mathfrak{l}}
\newcommand{\frm}{\mathfrak{m}}
\newcommand{\frn}{\mathfrak{n}}
\newcommand{\fro}{\mathfrak{o}}
\newcommand{\frp}{\mathfrak{p}}
\renewcommand{\frq}{\mathfrak{q}}
\newcommand{\frr}{\mathfrak{r}}
\newcommand{\frs}{\mathfrak{s}}
\newcommand{\frt}{\mathfrak{t}}
\newcommand{\fru}{\mathfrak{u}}
\newcommand{\frv}{\mathfrak{v}}
\newcommand{\frw}{\mathfrak{w}}
\newcommand{\frx}{\mathfrak{x}}
\newcommand{\fry}{\mathfrak{y}}
\newcommand{\frz}{\mathfrak{z}}
%
\newcommand{\bn}{\bold{n}}

% To-Do command
\newcommand{\todo}[1]{%
	{\textcolor{red}{\textbf{To-Do:} #1}}
}

% question command
\newcommand{\question}[1]{%
	{\textcolor{orange}{\textbf{Question:} #1}}
}

%% corrections
\newcommand{\correctb}[1]{{\textcolor{blue}{#1}}}
\newcommand{\correctr}[1]{{\textcolor{red}{#1}}}
\newcommand{\correcto}[1]{{\textcolor{orange}{#1}}}

% nice caption for images in tabuar
\newcommand{\mySubCaption}[2]{{\footnotesize ({#1}) {#2}}}

% kappa, and gamma bar
\newcommand{\stretchedStrain}[1]{\bar{#1}}

% convex conjugate
\newcommand{\ConvConj}[1]{{#1}^\star}

\begin{document}
	
\begin{center}
	\textbf{\large Additional instructions for reports in the Computational Dynamics group}
	
	\qquad
	
	Marco Herrmann, 
	Simon R. Eugster

	\today
	
\end{center}


\section*{Introduction}
%
This \LaTeX-file gives additional instructions how we expects students to write thier report within the Computational Dynamics group. It is an extension on the Dynamics \& Control template. Herein, we give help on writing down equations and using the provided commands, making mechanical drawings in inkscape, and creating figures with pgf plots.
%
\section{Writing equations and usefull commands}
%
Extending the nomenclature of the template, we use lowercase letters to indicate scalar variables, i.e., $c \in \mR$, bold lowercase letters to indicate vector quantities, i.e., $\vv \in \mR^3$ and bold uppercase letters to indicate matrix quantites, i.e., $\vA \in \mR^{3 \times 3}$. Since these are very common in mechanical reports, the file \textbf{header.tex} contains a lot of usefull commands to write down these bold characters by writing $\backslash$vv or $\backslash$vA in these example. Also bold greek letters are included with a similar structure to make writing down equations even more simple. We can write for example 
%
\begin{equation}
	\vga = \dot{\vg}
	\textAND
	\vLa = \int \vla \diff[t]
	\, .
\end{equation}
%
Also have a look at the other commands, allowing you to write calligraphic and fracture letters if needed. 

During the process of writing the report, you might indicate some changes in the text, or you want to put down a placeholder or a question directly. You can do so by using \correctb{$\backslash$correctb}, \correctr{$\backslash$correctr}, \correcto{$\backslash$correcto}, \todo{$\backslash$todo} and \question{$\backslash$question}.

\section{Drawing in inkscape}
%
For drawing, we advice you to use only the following linewidths:
%
\begin{center}

\begin{tabular}{ccccc}
	\toprule
	Very thick 
		& Thick
		& Thin
		& Very thin \\[2mm]
	0.6 mm 
		& 0.4 mm 
		& 0.2 mm
		& 0.1 mm \\
	\bottomrule
\end{tabular}
\end{center}
%
Mainly, thick and thin lines should be drawn. For arrows, the angle at the tip should be 15$^\circ$. When mechanical models are drawn, \textbf{thick} lines (0.4 mm) are used for body contours, forces and moments, while \textbf{thin} lines (0.2 mm) are used for coordinate systems, geometric measurements, hatching, guide lines, springs, dampers, gravitational acceleration. An example of a drawing of a mechanical system can be found in Figure~\ref{fig:AFM}.
%
\subsection{Adding text on drawings}
%
To finish the drawing, text containing information of geometric measures, vectors, mases, stiffness $\dots$ must be added. To do so, one can either use the Inkscape extension TexText, or write the text with the default text tool and use special export settings. TexText completely works in Inkscape, which means that the commands are not directly available. Furthermore, the text can be distrubed or scaled and a different font can be used without noticing. A benifit of TexText is the immidiat visibility of the inserted text. In contrast to this, the defualt text tool allows to write text using the math environment. For the export, one needs to choose under \textit{Text output options} the option \textit{Omit text in PDF and create LaTeX file}, which produces than a name.pdf file without any labels and a name.pdf\_tex file containing information where and what labels to put on drawing. With this method, the text on the drawing is compiled in the end together with the rest of the document. 
%
\begin{figure}
	\centering
	\input{graphics/AFM.pdf_tex}
	\caption{Mechanical model of the atomic force microscop with the excitations $a(t)$ and $\gamma(t)$. \question{How to put this: The caption of a figure should always classify to which experiment/simulation/$\dots$ it belongs.}}
	\label{fig:AFM}
\end{figure}
%
\section{Creating plots}
%
In case of a lot of plots and figures, it might be benefitial to create each figure in a sperate file to keep the main file clean. The individual files can then be included in the main report using the \textit{$\backslash$input} command. Since the possibilities to use tikz and pgfplots are very large, this document will just provide a simple example in Figure~\ref{fig:hermite_interpolation}. For explanations and more information we refer to the corresponding manuals. The python code to produce the data used in Figure~\ref{fig:hermite_interpolation} is also given with the files of this document. Therein, you can see which numpy command is used to save the results in this structure. \question{Shall we give advice on the plot line width? The widths from the INM template are given on the next side. I would recommend to set the linewidth once in \textit{pgfplotsset}, and always activate the grid.} \todo{Increase plot hieght if possible in the end!} \question{How to bring the two plots next to each other? Minipage? Subfigure?}

\newcommand{\plotPath}{how_to_export/results.csv}
\begin{figure}[!b]%
    \centering%
    \tikzsetnextfilename{cubic_hermite/position}%
    \begin{tikzpicture}%
        \begin{axis}[
            title={Position},
            height=0.25\textheight,
            width=0.55\textwidth,
            xlabel=$s$,
            grid=major,
            ymin=-0.9,
            ymax= 1.9,
            legend style={at={(0,1)}, anchor=north west},
        ]
        \addplot[red, solid]
            table [x=s, y=r_OP_x, col sep=comma]%
            {\plotPath};
        \addplot[green, solid]% 
            table [x=s, y=r_OP_y, col sep=comma]%
            {\plotPath};
        \addplot[blue, solid]% 
            table [x=s, y=r_OP_z, col sep=comma]%
            {\plotPath};
        %
        \addlegendentry{${}_I \vr_{OP_x}$}%
        \addlegendentry{${}_I \vr_{OP_y}$}%
        \addlegendentry{${}_I \vr_{OP_z}$}%
        \end{axis}
    \end{tikzpicture}%
    \hfill%
    \tikzsetnextfilename{cubic_hermite/tangent}%
    \begin{tikzpicture}%
        \begin{axis}[
            title={Tangent},
            height=0.25\textheight,
            width=0.55\textwidth,
            xlabel=$s$,
            grid=major,
            ymin=-1.1,
            ymax= 1.75,
            legend style={at={(0,1)}, anchor=north west},
        ]
        \addplot[red, solid]
            table [x=s, y=t_x, col sep=comma]%
            {\plotPath};
        \addplot[green, solid]% 
            table [x=s, y=t_y, col sep=comma]%
            {\plotPath};
        \addplot[blue, solid]% 
            table [x=s, y=t_z, col sep=comma]%
            {\plotPath};
        %
        \addlegendentry{${}_I \vt_x$}%
        \addlegendentry{${}_I \vt_y$}%
        \addlegendentry{${}_I \vt_z$}%
        \end{axis}
    \end{tikzpicture}%
    \caption{
        Cubic Hermite interpolation with non-uniform elements: The left plot shows the components of the position ${}_I\vr_{OP}$ along the arclength $s$ and the right plot shows the components tangent vector ${}_I\vt = {}_I \vr_{OP}^\prime$.
    }
    \label{fig:hermite_interpolation}
\end{figure}
\let\plotPath\undefined%%

\cleardoublepage
\subsection*{Computed lines}
\begin{itemize}
	\item Coordinate systems: 0.2 mm
	\item additional measurements: 0.2 mm
	\item individual functions: 0.4 mm
	\item sets of functions: 0.2 mm, to highlight individual ones: 0.4 mm
	\item coordinate grid: 0.1 mm (if neccessary)
\end{itemize}
%
\end{document} 
